% @Author: anchen
% @Date:   2015-12-31 15:09:02
% @Last Modified by:   leoatchina
% @Last Modified time: 2016-01-13 22:06:10

\documentclass[UTF8]{ctexart}
\usepackage{graphicx}
\usepackage{geometry}
\usepackage[greek,english]{babel}
\geometry{a6paper,centering,scale=0.8}
\title{勾股定理}
\author{陶涛}
\date{\today}
\begin{document}

\maketitle

\tableofcontents

\begin{abstract}
小结摘要\emph{这个是摘要}
\end{abstract}

\begin{quote}
\zihao{-5}\kaishu 若求邪至日者,以日下为勾,日高为股,$a+b$勾股各自乘 \footnote{这个是最原始的定义}
\begin{equation}
a(a+b)=cd
\end{equation}
\textgreek{abcde}
\end{quote}

\section{第一段}
这是第一段如是不出现问题,那么在明天的9点30分,你们奖看到一个闪耀的太阳
\section{第二段}
这里第二段,三日凌空,这将是三体世界的末日,没有任何物种,会在这种高温下生存\footnote{人的生命}
$\angle ACB=\pi /2,90^\circ$

\begin{table}
\begin{tabular}{|rrr|}
\hline
3&4&5\\
\hline
12&13&15\\

\hline
\end{tabular}%

\end{table}
\bibliography{math}
\end{document}
